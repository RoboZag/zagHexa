\documentclass[a4paper]{article}

%% Language and font encodings
\usepackage[english]{babel}
\usepackage[utf8x]{inputenc}
\usepackage[T1]{fontenc}

%% Sets page size and margins
\usepackage[a4paper,top=3cm,bottom=2cm,left=3cm,right=3cm,marginparwidth=1.75cm]{geometry}

%% Useful packages
\usepackage{amsmath}
\usepackage{graphicx}
\usepackage[colorinlistoftodos]{todonotes}
\usepackage[colorlinks=true, allcolors=blue]{hyperref}

\title{Chapter :     Simulation}


\begin{document}
\maketitle

\section{Introduction}
The first phase of robot manufacturing is its design and modeling. We can design and model the 
robot using CAD tools such as Solid Works, Blender, and so on. One of the main purposes 
of modeling robot is simulation.
\\The robotic simulation tool can check the critical flaws in the robot design and can confirm the 
working of the robot before it goes to the manufacturing phase.
\\The virtual robot model must have all the characteristics of real hardware, the shape of robot 
may or may not look like the actual robot but it must be an abstract, which has all the physical 
characteristics of the actual robot.
\\If we are planning to create the 3D model of the robot and simulate using ROS, you need to learn 
about some ROS packages which helps in robot designing.
\\ ROS has a standard meta package for designing, and creating robot models called robot model, 
which consists of a set of packages called urdf, robot state publisher and so on. 
These packages help us create the 3D robot model description with the exact characteristics of the 
real hardware.
\\ In this chapter, we will cover the following topics:
\begin{enumerate}
\item ROS packages for robot modeling

\item    Understanding robot modeling using URDF

\item    Creating the ROS package for the robot description

\item    Creating our first URDF model

\item    Explaining the URDF code
\item   Understanding robot modeling using xacro

\item     Creating our first Xacro model

\item    Explanation first Xacro model

\item  Conversion of xacro to URDF
\end{enumerate}

\section{Get deep into those topics}

\subsection{ROS packages for robot modeling}
ROS provides some good packages that can be used to build 3D robot models. In this
section, we will discuss some of the important ROS packages that are commonly used to
build robot models:
\\robot model: ROS has a meta package called robot model, which contains important 
packages that 
help build the 3D robot models. We can see all the important packages inside this meta-
package:
\\urdf: One of the important packages inside the robot model meta package is urdf. The 
URDF package contains a C++ parser for the Unified Robot Description Format (URDF),
which is an XML file to represent a robot model.
\\\\ We can define a robot model, sensors, and a working environment using URDF and 
can parse it using URDF parsers.
\\We can only describe a robot in URDF that has a tree-like 
structure in its links, that is, the robot will have rigid links and will be connected 
using joints. Flexible links can't be represented using URDF.
\\ The URDF is composed using special XML tags and we can parse these XML tags using 
parser programs for further processing. We can work on URDF modeling in the upcoming 
sections.
\\\\\textbf{joint state publisher}: This tool is very useful while designing robot 
models using URDF.
\\This package contains a node called joint state publisher, which reads the robot 
model description, finds all joints, and publishes joint values to all non fixed 
joints 
using GUI sliders. 
\\The user can interact with each robot joint using this tool and can visualize using 
RViz.
\\While designing URDF, the user can verify the rotation and translation of each 
joint using this tool. 
\\\\\textbf{kdl parser}: Kinematic and Dynamics Library (KDL) is an ROS package that 
contains parser tools to build a KDL tree from the URDF representation. The kinematic 
tree can be used to publish the joint states and also to forward and inverse 
kinematics of the robot.
\\\\\textbf{robot state publisher}: This package reads the current robot joint states 
and publishes 
the 3D poses of each robot link using the kinematics tree build from the URDF. The 3D 
pose of the robot is published as ROS tf (transform). ROS tf publishes the 
relationship 
between coordinates frames of a robot.
\\\\\textbf{xacro}: Xacro stands for (XML Macros) and we can define how xacro is equal 
to URDF plus add-ons. It contains some add-ons to make URDF shorter, readable, and can 
be used for building complex robot descriptions. We can convert xacro to URDF at any 
time using some ROS tools. We will see more about xacro and its usage in the upcoming 
sections.

\subsection{Understanding robot modeling using URDF}

We have discussed the urdf package. In this section, we will look further at the URDF XML tags, which help to model the robot. We have to create a file and write the relationship between each link and joint in the robot and save the file with the .urdf extension.
\\The URDF can represent the kinematic and dynamic description of the robot, visual representation of the robot, and the collision model of the robot.
\\\\The following tags are the commonly used URDF tags to compose a URDF robot model:
\\ \textbf{link}: The link tag represents a single link of a robot. Using this tag, we can model a robot link and its properties. The modeling includes size, shape, color, and can even import a 3D mesh to represent the robot link. We can also provide dynamic properties of the link such as inertial matrix and collision properties.
\\
The syntax is as follows:

<link name="<name of the link>">

<inertial>...........</inertial>

  <visual> ............</visual>

  <collision>..........</collision>

</link>
\\\\The following is a representation of a single link. The Visual section represents 
the real link of the robot, and the area surrounding the real link is the Collision 
section. The Collision section encapsulates the real link to detect collision before 
hitting the real link.
\\\\figure s1 is here: caption :Visualization of a URDF link \\
\\\textbf{joint}: The joint tag represents a robot joint. We can specify the kinematics and dynamics of the joint and also set the limits of the joint movement and its velocity. The joint tag supports the different types of joints such as revolute, continuous, prismatic,fixed, floating, and planar.
\\\\The syntax is as follows:\\\\

<joint name="<name of the joint>">

  <parent link="link1"/>

  <child link="link2"/>

  <calibration .... />

  <dynamics damping ..../>

  <limit effort .... />

</joint>
\\\\A URDF joint is formed between two links; the first is called the Parent link and the second is the Child link. The following is an illustration of a joint and its link:
\\\\figure s2 is here: caption :Visualization of a URDF joint \\
\\\\\textbf{robot}: This tag encapsulates the entire robot model that can be represented using URDF. Inside the robot tag, we can define the name of the robot, the links, and the joints of the robot.
\\\\The syntax is as follows:\\

<robot name="<name of the robot>"

  <link>  ..... </link>

  <link> ...... </link>

  <joint> ....... </joint>

  <joint> ........</joint>

</robot>
\\\\A robot model consists of connected links and joints. Here is a visualization of the robot model:
\\\\figure s3 is here: caption :Visualization of a robot model having joints and links \\
 \\\\\textbf{gazebo}: This tag is used when we include the simulation parameters of the Gazebo simulator inside URDF. We can use this tag to include gazebo plugins, gazebo material properties, and so on. The following shows an example using gazebo tags:

·          <gazebo reference="link 1">

·            <material>Gazebo/Black</material>

 </gazebo>

"We can find more URDF tags at http://wiki.ros.org/urdf/XML."
\subsection{Creating the ROS package for the robot description}
\textbf{TO BE CONTINUED}



\bibliographystyle{alpha}
\bibliography{sample}

\end{document}