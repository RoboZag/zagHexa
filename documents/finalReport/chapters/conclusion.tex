This paper presents a system description and the main aspects related to the design, construction, and implementation of six-leg robot named ZagHexa. The robot is a legged robot for search and rescue missions. It benefits form the reliability of its legged locomotion with the flexibility and versatility required to operate in different types of surface. The robot was constructed and tested to walk using tripod, wave and ripple gaits, can rotate and it is equipped with different sensors.

The robot was tested on different surfaces and in rugged terrain. The repeatability of the robot movement as well as the sensor system was also tested.	 These features are mainly achieved due to its original movement that make it deal with different surfaces. Additionally, its shape and weight give it more stability, and its ability to continue with its moving and sensing capabilities after collisions or even small falls. \\ 
However, more tests and experiments to improve and validate the design and sensor performance are to be carried out to optimize the system performance.
Finally, we are working on tackling some issues should to have fully autonomous operation and integration into a heterogeneous system. To make the integration of ZagHexa into different missions easier, an effort is being carried to provide it with a standard connectivity over the ROS framework.