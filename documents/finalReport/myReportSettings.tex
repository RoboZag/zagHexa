%!TEX TS-program = pdflatex
%!TeX encoding = UTF-8
%!TeX spellcheck = en_US
%!TeX root = userManual.tex
%!TeX author = Dr.Ing. Mohammed Nour Abdelgwad Ahmed
%!TeX version = 1.5
%!TeX date = 2017.03.10

\documentclass[%
DIV=12,
abstract=on
%a5paper,                 % alle weiteren Papierformat einstellbar
%landscape,              % Querformat
%10pt,                       % Schriftgröße (12pt, 11pt (Standard))
%BCOR1cm,               % Bindekorrektur, bspw. 1 cm
%DIVcalc,                  % führt die Satzspiegelberechnung neu aus scrguide s.2.4
%twoside,                 % Doppelseiten
%twocolumn,            % zweispaltiger Satz
%halfparskip*,           % Absatzformatierung s. scrguide 3.1
%headsepline,           % Trennline zum Seitenkopf
%footsepline,            % Trennline zum Seitenfuß
%titlepage,                % Titelei auf eigener Seite
%normalheadings ,    % Überschriften etwas kleiner (smallheadings)
%idxtotoc,                 % Index im Inhaltsverzeichnis
%liststotoc,               % Abb.- und Tab.verzeichnis im Inhalt
%bibtotoc,                 % Literaturverzeichnis im Inhalt
%abstracton,              % Überschrift über der Zusammenfassung an
%leqno,                      % Nummerierung von Gleichungen links
%fleqn,                       % Ausgabe von Gleichungen linksbündig
%draft                        % überlangen Zeilen in Ausgabe gekennzeichnet
]
{scrartcl} %scrreprt

% Deutsche Anpassungen %
%\usepackage[ngerman]{babel}
\usepackage[T1]{fontenc}
\usepackage[utf8]{inputenc}
\usepackage{lmodern} 	%Type1-Schriftart für nicht-englische Texte
\usepackage{amssymb} %provides various useful mathematical symbols
\usepackage{amsthm}  %provides extended theorem environments
\usepackage{amsmath} %provides the align environment
\usepackage{xcolor}
\usepackage[bookmarks,
  bookmarksopen=false,
  bookmarksnumbered=true,
  pdftex,pdfhighlight=/N,
  linkcolor=blue!60!black,
  urlcolor=blue!60!black,
  colorlinks=true,
  citecolor=green!20!black,
  pdftitle={Steering Control for Autonomous Path Tracking},
  pdfsubject={EO2 Path Following Documentation},  % insert subtitle
  pdfkeywords={terramechanics, testbed, force measurement, soil contact interactions, impedance control, legged robots, simulation},
  pdfauthor={Dr.Ing. Mohammed Nour Abdelgwad Ahmed}]{hyperref}

% Packages für Grafiken & Abbildungen %
\usepackage{graphicx} 	%Zum Laden von Grafiken
%\usepackage{subfig} 	%Teilabbildungen in einer Abbildung
\usepackage{wrapfig}

% Bibliographiestil %
%\usepackage{natbib}

%------------------------------------------------------------------------------
\usepackage[automark,headsepline]{scrlayer-scrpage}
\clearpairofpagestyles
\cfoot[\pagemark]{\pagemark}
\lehead{\headmark}
\rohead{\headmark}
\pagestyle{scrheadings}

%\setkomafont{pagehead}{\normalfont\bfseries}
%\renewcommand*\pagemark{{\usekomafont{pagenumber}Page\nobreakspace\thepage}}
%\addtokomafont{pageheadfoot}{\upshape}
%------------------------------------------------------------------------------


  \usepackage{tikz} %for the revision No. block on the margin of title (first) page
%\usetikzlibrary{shapes}

\definecolor{thegrey} {gray}{0.5}
\definecolor{theshade}{gray}{0.94}
\definecolor{theframe}{gray}{0.75}

\usepackage{myboxedtheorem}
%\newboxedtheorem[options]{theo}{Theorem}{thCounter}
%boxcolor = black, titleboxcolor = black, background = white, titlebackground = white, titlecolor = black, thcounter=, size = .9\textwidth
%\newboxedtheorem[boxcolor=blue!20, background=blue!5, titlebackground=blue!20, titleboxcolor = blue!20]{theo}{Theorem}{anything}
\newboxedtheorem[boxcolor=theframe, background=theshade, titlebackground=thegrey,%
                 titlecolor = white, titleboxcolor = thegrey, size = 0.8\textwidth]{myColoredBox}{}{}

\newboxedtheorem[boxcolor=pink, titleboxcolor = pink, titlebackground=red!20, background=red!5]{mytheo}{}{}
%------------------------------------------------------------------------------
\setkomafont{title}{\normalfont\bfseries\small} %to set the tile font (I hate the default one ;))

\graphicspath{{./figures/}}
\pagestyle{empty} %Keine Kopf-/Fusszeilen auf den ersten Seiten.
%==============================================================================